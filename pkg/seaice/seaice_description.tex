%%
%%  $Header: /home/jahn/src/cvs2git/MITgcm/20170915-2/gcmpack-all-patch/MITgcm/pkg/seaice/seaice_description.tex,v 1.1 2004/05/05 07:15:41 dimitri Exp $
%%  $Name:  $
%%

\chapter{Dynamic Thermodynamic Seaice Package}
  
Package ``seaice'' provides a dynamic and thermodynamic interactive sea-ice
model.


Sea ice covers up to $30\times 10^6$ km$^2$ of the ocean's surface
(almost 10\%).  Because of lack of data, meteorological fields have
large uncertainties in polar regions.  The inclusion of a sea ice
model will permit ECCO to make use of satellite observations over
ice-covered oceans.  For example, SSM/I ice motion may be a better
estimate of surface stress than that provided by NCEP or ECMWF.
Another motivation for including sea ice is that global estimates of
atmospheric fluxes over the ocean are {\em not} possible without a sea
ice model because of huge storage and transport sea ice terms, e.g.,
latent heat of fusion and freshwater transport.  Finally, a complete
global ocean state estimation, the central goal of ECCO, requires
inclusion of the Arctic Ocean.  Proper representation of polar regions
in the ECCO products will make it possible to address a host of
interesting and important science questions pertaining to
oceanographic links in polar-subpolar interactions.

A sea ice model is now available for the MIT GCM and ready for
inclusion in the first public release of the model.  The sea ice code
is robust (53-year global integration with NCEP forcing), sensible
(results already compare favorably with data prior to any tuning or
assimilation), clean (a single entry point and relatively trouble-free
parsing by the adjoint model compiler), and parallelized (the 53-year
integration used 32 processors).  Herein we discuss sea-ice model
characteristics, its numerical implementation, some prelinary
results, and remaining challenges.

\section{Sea-Ice Model Description}

Sea-ice model thermodynamics are based on Hibler \cite{hib80}, that is, a
2-category model that simulates ice thickness and concentration.  Snow is
simulated as per Zhang et al. \cite{zha98a}.  Although recent years have seen
an increased use of multi-category thickness distribution sea-ice models for
climate studies, the Hibler 2-category ice model is still the most widely used
model and has resulted in realistic simulation of sea-ice variability on
regional and global scales.  Being less complicated, compared to
multi-category models, a 2-category model will permit easier application of
adjoint model optimization methods.

Note, however, that the Hibler 2-category model and its variants use a
so-called zero-layer thermodynamic model to estimate ice growth and
decay.  The zero-layer thermodynamic model assumes that ice does not
store heat and, therefore, tends to exaggerate the seasonal
variability in ice thickness.  This exaggeration can be significantly
reduced by using Semtner's \cite{sem76} three-layer thermodynamic
model that permits heat storage in ice.  Recently, the three-layer
thermodynamic model has been reformulated by Winton \cite{win00}.  The
reformulation improves model physics by representing the brine content
of the upper ice with a variable heat capacity.  It also improves
model numerics and consumes less computer time and memory.  We plan to
adapt and make use of Winton's thermodynamic sea-ice model in our
study.

The ice dynamics models that are most widely used for large-scale
climate studies are the viscous-plastic (VP) model \cite{hib79}, the
cavitating fluid (CF) model \cite{fla92}, and the
elastic-viscous-plastic (EVP) model \cite{hun97}.  Compared to the VP
model, the CF model does not allow ice shear in calculating ice
motion, stress, and deformation.  EVP models approximate VP by adding
an elastic term to the equations for easier adaptation to parallel
computers.  Because of its higher accuracy in plastic solution and
relatively simpler formulation, compared to the EVP model, we decided
to use the VP model as the dynamic component of our ice model.  To do
this we extended the alternating-direction-implicit (ADI) method of
Zhang and Rothrock \cite{zha00} for use in a parallel configuration.

\section{Numerical Implementation}

The sea ice model is implemented as a package and follows standard MIT
GCM package structure and guidelines.  To improve portability to all
existing and future versions of the MIT GCM, the sea ice package has a
single entry point: it replaces the subroutine that loads external
forcing fields.  All sea-ice input, output, checkpointing, and pickup
operations as well as the dynamic and thermodynamic components take
place during this single call.  Furthermore, contrary to existing
practice, the sea ice model does not modify surface temperature and
salinity directly.  Instead the interaction with the MIT GCM is
through the computation of equivalent net fluxes at the ice-ocean
interface.  This facilitates interfacing with other MIT GCM packages,
for example, the mixed layer parameterization.

The sea ice model requires the following input fields: 10-m winds, 2-m air
temperature and specific humidity, downward longwave and shortwave radiations,
precipitation, evaporation, and river and glacier runoff.  The sea ice model
also requires surface temperature from the ocean model and third level
horizontal velocity which is used as a proxy for surface geostrophic
velocity.  Output fields are surface wind stress, evaporation minus
precipitation minus runoff, net surface heat flux, and net shortwave flux.
The sea-ice model is global: in ice-free regions bulk formulae are used to
estimate oceanic forcing from the atmospheric fields.

\section{Preliminary Results}

The coupled sea-ice ocean model was tested in a 53-year integration using the
same configuration as that of the ECCO 2$^\circ$ optimization, that is, an
80$^\circ$S--80$^\circ$N quasi-global configuration with 23 vertical levels.
KPP and GM mixing schemes were turned on.  Daily forcing (12-hourly for winds)
was from the NCEP 1948--2000 reanalysis.  Overall, the resulting estimates of
sea ice extent compare favorably with passive microwave (SMMR, SSM/I) data
producing a realistic seasonal cycle and interannual variability prior to any
tuning or data assimilation.  There are of course residual differences, for
example, the model tends to underestimate extent of summer sea-ice around
Antarctica.  These differences will become the signal for the planned
assimilation of passive microwave data.

To understand the impact of sea ice on the large scale ocean circulation, a
second 53-year integration was carried out excluding the dynamic/thermodynamic
sea-ice model, but maintaining a minimum cap of $-1.8^\circ$ C in the surface
temperature.  Fig.~\ref{overturn} compares meridional overturning strength in
North Atlantic and Southern Oceans for the two 53-year integrations.  The
presence of sea ice in the model damps both the amplitude and variability of
overturning strength.  Although the causes for these large differences have
not yet been fully diagnosed, it is clear that the proper representation of
sea ice can have a huge impact on the large-scale model circulation.

\section{Discussion}

The sea ice model is an important step towards the goal of a truly {\em
global} ocean circulation estimate by the ECCO consortium.  But much work is
still needed.  This includes obtaining and testing the adjoint sea-ice model,
learning to use passive microwave and other sea ice data as constraints in the
optimization, and experimenting with effective sea ice control parameters, for
example, ice strength, drag coefficients, and surface albedo.  It is proposed
that these tasks be under-
%\begin{figure}[t]
%{\psfig{file=FORECCO1.eps,width=3.2in}}
%\vspace{-.7cm}
%\caption{Impact of Sea Ice on Ocean Circulation.}
%\label{overturn}
%\end{figure}
%\vspace{.3cm}
\noindent taken in the older, quasi-global, 2$^\circ$ ECCO configuration
before being ported to the current higher resolution one.  Another important
task is the inclusion of the Arctic Ocean, either through the elegant, but
still experimental, cubed-sphere configuration that is being developed by
A. Adcroft or through a more conventional approach, for example, by displacing
the North Pole singularity over a land location or by using a nested-grid.
